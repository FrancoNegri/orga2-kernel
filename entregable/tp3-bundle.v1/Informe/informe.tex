\documentclass[a4paper]{article}
\usepackage[spanish]{babel}
\usepackage[utf8]{inputenc}
\usepackage{charter}   % tipografia
\usepackage{graphicx}
%\usepackage{makeidx}
\usepackage{paralist} %itemize inline

%\usepackage{float}
%\usepackage{amsmath, amsthm, amssymb}
%\usepackage{amsfonts}
%\usepackage{sectsty}
%\usepackage{charter}
%\usepackage{wrapfig}
%\usepackage{listings}
%\lstset{language=C}


\input{codesnippet}
\input{page.layout}
% \setcounter{secnumdepth}{2}
\usepackage{underscore}
\usepackage{caratula}
\usepackage{url}


% ******************************************************** %
%              TEMPLATE DE INFORME ORGA2 v0.1              %
% ******************************************************** %
% ******************************************************** %
%                                                          %
% ALGUNOS PAQUETES REQUERIDOS (EN UBUNTU):                 %
% ========================================
%                                                          %
% texlive-latex-base                                       %
% texlive-latex-recommended                                %
% texlive-fonts-recommended                                %
% texlive-latex-extra?                                     %
% texlive-lang-spanish (en ubuntu 13.10)                   %
% ******************************************************** %



\begin{document}


\thispagestyle{empty}
\materia{Organización del Computador II}
\submateria{Segundo Cuatrimestre de 2014}
\titulo{Trabajo Práctico III}
\subtitulo{El Kernel contraataca}
\integrante{Alejandro Mignanelli}{609/11}{minga_titere@hotmail.com}
\integrante{Franco Negri}{893/13}{franconegri2004@hotmail.com}
\integrante{Federico Suárez}{610/11}{elgeniofederico@gmail.com}

\maketitle
\newpage

\thispagestyle{empty}
\vfill
\begin{abstract}
En el presente trabajo se describe el desarrolo del Kernel desarrollado para una arquitectura intel de 32-bits, así como el manejo de paginación, manejo de tareas, interrupciones y todo lo referente al manejo de un pequeño sistema operativo.

\end{abstract}

\thispagestyle{empty}
\vspace{3cm}
\tableofcontents
\newpage

%\normalsize
\newpage

\section{Objetivos generales}

\clearpage

\section{Preambulo}

\clearpage

\section{GDT}
\input{GDT/gdt.tex}

\clearpage

\section{Interrupciones} 
\input{Interrupciones/int.tex}

\clearpage

\section{MMU}
\input{MMU/mmu.tex}

\clearpage

\section{Paginación}
\input{Paginacion/pag.tex}

\clearpage

\section{Scheduler}
\input{Scheduler/scheduler.tex}


\end{document}

