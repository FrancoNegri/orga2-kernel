\documentclass[a4paper]{article}
\usepackage[spanish]{babel}
\usepackage[utf8]{inputenc}
\usepackage{charter}   % tipografia
\usepackage{graphicx}
%\usepackage{makeidx}
\usepackage{paralist} %itemize inline

%\usepackage{float}
%\usepackage{amsmath, amsthm, amssymb}
%\usepackage{amsfonts}
%\usepackage{sectsty}
%\usepackage{charter}
%\usepackage{wrapfig}
%\usepackage{listings}
%\lstset{language=C}


\usepackage{color} % para snipets de codigo coloreados
\usepackage{fancybox}  % para el sbox de los snipets de codigo

\definecolor{litegrey}{gray}{0.94}

% \newenvironment{sidebar}{%
% 	\begin{Sbox}\begin{minipage}{.85\textwidth}}%
% 	{\end{minipage}\end{Sbox}%
% 		\begin{center}\setlength{\fboxsep}{6pt}%
% 		\shadowbox{\TheSbox}\end{center}}
% \newenvironment{warning}{%
% 	\begin{Sbox}\begin{minipage}{.85\textwidth}\sffamily\lite\small\RaggedRight}%
% 	{\end{minipage}\end{Sbox}%
% 		\begin{center}\setlength{\fboxsep}{6pt}%
% 		\colorbox{litegrey}{\TheSbox}\end{center}}

\newenvironment{codesnippet}{%
	\begin{Sbox}\begin{minipage}{\textwidth}\sffamily\small}%
	{\end{minipage}\end{Sbox}%
		\begin{center}%
		\vspace{-0.4cm}\colorbox{litegrey}{\TheSbox}\end{center}\vspace{0.3cm}}



\usepackage{fancyhdr}
\pagestyle{fancy}

%\renewcommand{\chaptermark}[1]{\markboth{#1}{}}
\renewcommand{\sectionmark}[1]{\markright{\thesection\ - #1}}

\fancyhf{}

\fancyhead[LO]{Sección \rightmark} % \thesection\ 
\fancyfoot[LO]{\small{Alejandro Mignanelli, Franco Negri, Federico Suárez}}
\fancyfoot[RO]{\thepage}
\renewcommand{\headrulewidth}{0.5pt}
\renewcommand{\footrulewidth}{0.5pt}
\setlength{\hoffset}{-0.8in}
\setlength{\textwidth}{16cm}
%\setlength{\hoffset}{-1.1cm}
%\setlength{\textwidth}{16cm}
\setlength{\headsep}{0.5cm}
\setlength{\textheight}{25cm}
\setlength{\voffset}{-0.7in}
\setlength{\headwidth}{\textwidth}
\setlength{\headheight}{13.1pt}

\renewcommand{\baselinestretch}{1.1}  % line spacing


% \setcounter{secnumdepth}{2}
\usepackage{underscore}
\usepackage{caratula}
\usepackage{url}


% ******************************************************** %
%              TEMPLATE DE INFORME ORGA2 v0.1              %
% ******************************************************** %
% ******************************************************** %
%                                                          %
% ALGUNOS PAQUETES REQUERIDOS (EN UBUNTU):                 %
% ========================================
%                                                          %
% texlive-latex-base                                       %
% texlive-latex-recommended                                %
% texlive-fonts-recommended                                %
% texlive-latex-extra?                                     %
% texlive-lang-spanish (en ubuntu 13.10)                   %
% ******************************************************** %



\begin{document}


\thispagestyle{empty}
\materia{Organización del Computador II}
\submateria{Segundo Cuatrimestre de 2014}
\titulo{Trabajo Práctico III}
\subtitulo{El Kernel contraataca}
\integrante{Alejandro Mignanelli}{609/11}{minga_titere@hotmail.com}
\integrante{Franco Negri}{893/13}{franconegri2004@hotmail.com}
\integrante{Federico Suárez}{610/11}{elgeniofederico@gmail.com}

\maketitle
\newpage

\thispagestyle{empty}
\vfill
\begin{abstract}
En el presente trabajo se describe el desarrolo del Kernel desarrollado para una arquitectura intel de 32-bits, así como el manejo de paginación, manejo de tareas, interrupciones y todo lo referente al manejo de un pequeño sistema operativo.

\end{abstract}

\thispagestyle{empty}
\vspace{3cm}
\tableofcontents
\newpage

%\normalsize
\newpage

\section{Objetivos generales}

El objetivo de este trabajo practico, partiendo de un procesador intel de 32-bits, generar un kernel capaz de gestionar memoria entre diferentes tareas, correrlas de manera concurrente, y resolver las diferentes problematicas que puedan surgir al momento de ejecución.

Para ello utilizaremos las diversas herramientas que intel pone a nuestra disposicion en modo protegido: Usaremos segmentacion y paginación para controlar el privilegio con el que las tareas se ejecutarán.
Utilizaremos interrupciones del procesador que permitirán, tanto reaccionar de manera apropiada cuando se produzca un error en tiempo de ejecución, obtener input del teclado y gestionar un task manager que nos perimita ejecutar tareas de manera concurrente.

En las siguientes secciones se detallará de manera mas elaborada el funcionamiento e implementación de cada una de las partes principales del sistema.

\clearpage

\section{GDT}
\input{GDT/gdt.tex}

\clearpage

\section{Interrupciones} 
\input{Interrupciones/int.tex}

\clearpage

\section{MMU}
\input{MMU/mmu.tex}

\clearpage

\section{Paginación}
\input{Paginacion/pag.tex}

\clearpage

\section{Scheduler}
\input{Scheduler/scheduler.tex}


\end{document}

